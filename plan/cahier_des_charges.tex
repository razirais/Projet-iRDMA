\newcommand{\TITLE}{Cahier des Charges}

\documentclass[a4paper,10pt]{article}

\usepackage[french]{babel}
\usepackage[utf8]{inputenc}
\usepackage[T1]{fontenc}

\usepackage{amsmath}
\usepackage{amsfonts}
\usepackage{amssymb}

\usepackage{graphicx}
\usepackage{array}
\usepackage{pgf}
\usepackage{tikz}
\usepackage{listings}
\usepackage{eurosym}
\usepackage{listing}

\newcommand{\unitnb}[2]{\unit{\nombre{#1}}{#2}}
\newcommand{\hexa}[1]{$\mathrm{\textbf{0x}}{#1}$}
\definecolor{vert}{rgb}{0.2,0.6,0.4} 
\definecolor{gris}{rgb}{0.4,0.4,0.4} 

\newcommand{\lettrine}[2]{\hspace{5mm}\LARGE{#1}\normalsize{#2}}

\lstset{
	language=C,
	basicstyle=\footnotesize,
	numbers=left,
	numberstyle=\tiny,
	stepnumber=5,
	frame=single,
	captionpos=t,
	breaklines=true,
	keywordstyle=\color{blue}\textbf,
	commentstyle=\color{vert}\textit,
	stringstyle=\color{gris}\texttt
}

\newenvironment{changemargin}[2]{\begin{list}{}{
	\setlength{\topsep}{0pt}
	\setlength{\leftmargin}{0pt}
	\setlength{\rightmargin}{0pt}
	\setlength{\listparindent}{\parindent}
	\setlength{\itemindent}{\parindent}
	\setlength{\parsep}{0pt plus 1pt}
	\addtolength{\leftmargin}{#1}
	\addtolength{\rightmargin}{#2}
}\item }{\end{list}}

%--------------------EVITE LES ORPHELINS-----------
\widowpenalty=10000
\clubpenalty=10000
\raggedbottom
%--------------------------------------------------

\makeatletter
\@addtoreset{chapter}{part}
\makeatother

%\renewcommand{\FrenchLabelItem}{\textbullet}

\textwidth = 15cm
\hoffset = -1.54cm
\voffset = -1cm
\headsep = 0cm
\headheight = 0cm
\topmargin = 1cm
\textheight = 22cm

%% Package pour faire des liens dynamiques entre les pdf
\usepackage[pdftex,
		    bookmarks         = true,       % Signets
		    bookmarksnumbered = true,       % Signets numérotés
		    pdfpagemode       = None,       % Signets fermé à l'ouverture
		    pdfstartview      = FitH,       % La page prend toute la largeur
		    colorlinks        = true,       % Liens en couleur
		    urlcolor          = cyan,				% Couleur des liens externes
		    pdfborder         = {0 0 0}   	% Style de bordure
    		]{hyperref}

\hypersetup{
    pdfauthor   = {Adrien Oliva, Gaëtan Harter},
    pdftitle    = {Projet OCAE},
    pdfsubject  = {Projet systême embarqué},
    pdfkeywords = {Projet SLE systême embarqué},
    pdfcreator  = {PDFLaTeX},
    pdfproducer = {PDFLaTeX}
}

\begin{document}

\begin{minipage}[c]{0.45\linewidth}
\rule{5cm}{1pt}

Gaëtan Harter

Adrien Oliva

\rule{5cm}{1pt}
\end{minipage} \hfill
\begin{minipage}[c]{0.45\linewidth}
\begin{flushright}
\vspace{5mm}\includegraphics[width=6cm]{../img/ensimag.png}

\footnotesize{\textsc{\'Ecole Nationale Sup\'erieur d'Informatique}}

\footnotesize{\textsc{ et de Math\'ematiques Appliqu\'ees}}

\footnotesize{681, rue de la Passerelle, 38400 Saint Martin d'Heres}
\end{flushright}
\end{minipage}

\vspace{2cm}

\begin{center}

\begin{LARGE}
Projet SLE
\end{LARGE}

\begin{LARGE}
Écriture de driver RocketIO
\end{LARGE}

\vspace{5mm}

\begin{Large}
\TITLE
\end{Large}

\rule{6cm}{2pt}
\end{center}

\vspace{25mm}

\begin{flushright}

\includegraphics[width=5cm]{../img/xilinx.png}
\includegraphics[width=5cm]{../img/virtex4.png}

\rule{11cm}{0.5pt}

\vspace{-3mm}
\rule{7cm}{0.5pt}

\vspace{-3mm}
\rule{5cm}{0.5pt}

\vspace{-3mm}
\rule{4cm}{0.5pt}

\vspace{5mm}
\href{mailto:gaetan.harter@ensimag.imag.fr}
{\texttt{gaetan.harter@ensimag.imag.fr}}

\href{mailto:adrien.oliva@ensimag.imag.fr}
{\texttt{adrien.oliva@ensimag.imag.fr}}
\end{flushright}

\vfill
\eject

\tableofcontents

\vfill
\eject





\section{Contrôleur RocketIO}

Le rocket IO est un contrôleur réseau couplé à un DMA. Il permet 
d'envoyer des paquets, d'en réceptionner comme un contrôleur classique
mais aussi de recevoir et de stocker des paquets directement en mémoire,
sans avoir à passer par le système d'exploitation.

\section{Driver}

Le but du projet est de réaliser la partie logiciel liant le contrôleur 
au système d'exploitation, un noyau linux 2.6.32, sur une carte Xilinx
Virtex 4.

Ce driver prendra la forme d'un module réseau pour un noyau linux. Il 
devra supporter l'envoi de paquets TCP/IP et la réception de paquets DDP/IP.
Ce dernier est un protocole qui permet la réception de paquets directement
en mémoire. 

La réception des paquets se fait dans un espace mémoire réservé au driver,
sous la forme d'un tampon circulaire. 

Pour une utilisation plus générale du driver, on implémentera également la 
réception TCP/IP et l'émission DDP/IP.

La communication entre le module, le matériel et le 
système d'exploitation s'effectue via des interruptions.

\section{Planning}

\begin{tabular}{ll}
\oldstylenums{21} sept. \oldstylenums{2010} & Choix du sujet \\
\oldstylenums{28} sept. \oldstylenums{2010} & Première approche du LDD \\
du \oldstylenums{4} au \oldstylenums{19} oct. \oldstylenums{2010} &
Compilation de l'environnement de cross-compilation \\
& Démonstration du driver sur machine nue et de l'environnement Xilinx \\
du \oldstylenums{24} oct au \oldstylenums{2} nov. \oldstylenums{2010} &
Lecture du Linux Device Driver \\
& Lecture des anciens rapports concernant le contrôleur Rocket IO \\
& Élaboration du cahier des charges \\
du \oldstylenums{2} au \oldstylenums{9} nov. \oldstylenums{2010} &
Allégement et portage du module préexistant pour noyau 2.4 jusqu'à \\
& compilation et fonctionnement sur carte (chargement et déchargement,\\
& sans réelles fonctionnalités) \\
du \oldstylenums{9} nov. au \oldstylenums{14} dec. \oldstylenums{2010} &
Ajout progressif des fonctionnalités du driver avec le banc de test associé\\
\end{tabular}

La fin du projet est consacré au débugage et à l'écriture de la documentation
même si ces deux étapes seront en partie réalisées au cours du développement.

\subsection*{Ajout de fonctionnalités}

Les fonctionnalités suivantes seront implantées les unes à la suite des autres~:

\begin{itemize}
\item chargement et déchargement du module
\item ouverture et fermeture de l'interface réseau
\item gestion du tampon circulaire
\item réception et envoi via le protocole TCP/IP
\item envoi de paquets DDP/IP
\item réception DDP/IP
\end{itemize}

\end{document}
