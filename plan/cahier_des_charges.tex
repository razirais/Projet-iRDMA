\newcommand{\TITLE}{Cahier des Charges}

\documentclass[a4paper,10pt]{article}

\usepackage[french]{babel}
\usepackage[utf8]{inputenc}
\usepackage[T1]{fontenc}

\usepackage{amsmath}
\usepackage{amsfonts}
\usepackage{amssymb}

\usepackage{graphicx}
\usepackage{array}
\usepackage{pgf}
\usepackage{tikz}
\usepackage{listings}
\usepackage{eurosym}
\usepackage{listing}

\newcommand{\unitnb}[2]{\unit{\nombre{#1}}{#2}}
\newcommand{\hexa}[1]{$\mathrm{\textbf{0x}}{#1}$}
\definecolor{vert}{rgb}{0.2,0.6,0.4} 
\definecolor{gris}{rgb}{0.4,0.4,0.4} 

\newcommand{\lettrine}[2]{\hspace{5mm}\LARGE{#1}\normalsize{#2}}

\lstset{
	language=C,
	basicstyle=\footnotesize,
	numbers=left,
	numberstyle=\tiny,
	stepnumber=5,
	frame=single,
	captionpos=t,
	breaklines=true,
	keywordstyle=\color{blue}\textbf,
	commentstyle=\color{vert}\textit,
	stringstyle=\color{gris}\texttt
}

\newenvironment{changemargin}[2]{\begin{list}{}{
	\setlength{\topsep}{0pt}
	\setlength{\leftmargin}{0pt}
	\setlength{\rightmargin}{0pt}
	\setlength{\listparindent}{\parindent}
	\setlength{\itemindent}{\parindent}
	\setlength{\parsep}{0pt plus 1pt}
	\addtolength{\leftmargin}{#1}
	\addtolength{\rightmargin}{#2}
}\item }{\end{list}}

%--------------------EVITE LES ORPHELINS-----------
\widowpenalty=10000
\clubpenalty=10000
\raggedbottom
%--------------------------------------------------

\makeatletter
\@addtoreset{chapter}{part}
\makeatother

%\renewcommand{\FrenchLabelItem}{\textbullet}

\textwidth = 15cm
\hoffset = -1.54cm
\voffset = -1cm
\headsep = 0cm
\headheight = 0cm
\topmargin = 1cm
\textheight = 22cm

%% Package pour faire des liens dynamiques entre les pdf
\usepackage[pdftex,
		    bookmarks         = true,       % Signets
		    bookmarksnumbered = true,       % Signets numérotés
		    pdfpagemode       = None,       % Signets fermé à l'ouverture
		    pdfstartview      = FitH,       % La page prend toute la largeur
		    colorlinks        = true,       % Liens en couleur
		    urlcolor          = cyan,				% Couleur des liens externes
		    pdfborder         = {0 0 0}   	% Style de bordure
    		]{hyperref}

\hypersetup{
    pdfauthor   = {Adrien Oliva, Gaëtan Harter},
    pdftitle    = {Projet OCAE},
    pdfsubject  = {Projet systême embarqué},
    pdfkeywords = {Projet SLE systême embarqué},
    pdfcreator  = {PDFLaTeX},
    pdfproducer = {PDFLaTeX}
}

\begin{document}

\begin{minipage}[c]{0.45\linewidth}
\rule{5cm}{1pt}

Gaëtan Harter

Adrien Oliva

\rule{5cm}{1pt}
\end{minipage} \hfill
\begin{minipage}[c]{0.45\linewidth}
\begin{flushright}
\vspace{5mm}\includegraphics[width=6cm]{../img/ensimag.png}

\footnotesize{\textsc{\'Ecole Nationale Sup\'erieur d'Informatique}}

\footnotesize{\textsc{ et de Math\'ematiques Appliqu\'ees}}

\footnotesize{681, rue de la Passerelle, 38400 Saint Martin d'Heres}
\end{flushright}
\end{minipage}

\vspace{2cm}

\begin{center}

\begin{LARGE}
Projet SLE
\end{LARGE}

\begin{LARGE}
Écriture de driver RocketIO
\end{LARGE}

\vspace{5mm}

\begin{Large}
\TITLE
\end{Large}

\rule{6cm}{2pt}
\end{center}

\vspace{25mm}

\begin{flushright}

\includegraphics[width=5cm]{../img/xilinx.png}
\includegraphics[width=5cm]{../img/virtex4.png}

\rule{11cm}{0.5pt}

\vspace{-3mm}
\rule{7cm}{0.5pt}

\vspace{-3mm}
\rule{5cm}{0.5pt}

\vspace{-3mm}
\rule{4cm}{0.5pt}

\vspace{5mm}
\href{mailto:gaetan.harter@ensimag.imag.fr}
{\texttt{gaetan.harter@ensimag.imag.fr}}

\href{mailto:adrien.oliva@ensimag.imag.fr}
{\texttt{adrien.oliva@ensimag.imag.fr}}
\end{flushright}

\vfill
\eject

\tableofcontents

\vfill
\eject





\section{Fonction du driver}

Le module s'articule en plusieurs parties.

\begin{itemize}
\item Chargement et déchargement du module
\item Gestionnaire d'interruptions
\item Fonctions d'envoi de paquet sur TCP/IP et sur DDP/IP
\item Fonctions de réception de paquet (dépend du protocole d'envoi du
paquet )
\end{itemize}

\subsection{Gestion du module}



\subsection{Gestionnaire d'interruptions}

\subsection{Envoi et réception de paquets}

L'envoi et la réception de paquet, que se soit TCP ou DDP utilise un ring
buffer. Un message TCP est entièrement stocké dans un ou plusieurs paquets
tandis que le message DDP contient une adresse mémoire pointant sur les
données du paquet.

\subsubsection{Transmission}




\section{Protocole DDP}



\end{document}
